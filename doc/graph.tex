\documentclass[a4paper,8pt]{article}
\usepackage[utf8]{inputenc}
\usepackage[brazil]{babel}
\usepackage{url}
%\usepackage[font=small],{caption}

% The following is needed in order to make the code compatible
% with both latex/dvips and pdflatex.
\usepackage{ifpdf}
\ifpdf
	\usepackage[pdftex]{graphicx}
	\DeclareGraphicsRule{*}{mps}{*}{}
\else
	\usepackage[dvips]{graphicx}
\fi

\newcommand{\smartmakeincludegraphics}[2][]{
	\ifpdf
		\includegraphics[#1]{#2.pdf}
	\else
		\includegraphics[#1]{#2.eps}
	\fi
}


\title{T2 modular}
\author{Marcelo Politzer, Raphael Bertoche, Gabriel Arrais}

\begin{document}

\maketitle

\tableofcontents
\section{Requisitos Funcionais}


\begin{enumerate}
\item Criar um usuário com uma UserID de até 15 caracteres, composta por
	[a-z;0-9;.-\_@]. (é acessado no menu de login.)

\item Logar com usuário já existente, que significa modificar o usuário
	corrente (nao é possível logar com usuário inexistente, é acessado no menu
	de login.)

\subsection{Requisitos que definem comandos}
\item Excluir perfil
	\newline\texttt{delme}

\item Editar perfil e interesses (Amizade, Trabalho, relacionamento aberto...)
	\newline\texttt{editme}

\item Procurar perfis de outros usuários filtrando por: interesse, amizade,
	idade, ID, nome
	\newline\texttt{search [-f] [-u userid] [-i interest] [-a minage-maxage]}
	\newline-f - mostra apenas amigos
\item Extinção do módulo Mensagem

\item Adicionar novo contato (incluir usuário nos contatos)
	\newline\texttt{addfriend userid}

\item Retirar contato (excluir usuário dos contatos)
	\newline\texttt{unfriend userid}

\item Enviar mensagens para mais de um destinatário dentre os usuários já
	adicionados.
	\newline\texttt{write [destinatario1] [destinatario2] ...}

\item Arquivar mais de uma mensagem na caixa de mensagens de cada usuário
	\newline\texttt{msglist []}
	\newline\texttt{msgread number}
	\newline\texttt{msgdel number}

\end{enumerate}

\section{Requisitos Não Funcionais}

\begin{enumerate}

	\item Portabilidade

	O projeto deve suportar compilação tanto utilizando os Gnu-Make e GCC quanto
	utilizando o Visual Studio e o GMake;

	\item Performance

	Nenhum comando pode demorar mais que 10ms por usuário registrado;

	\item Segurança

	As funções que lêem da entrada padrão devem estar protegidas de
	overflow de qualquer espécie;

	\item Privacidade

	Deve ser impossível de se ler mensagens alheias e de conhecer listas de
	amigos.

\end{enumerate}

\section{Diagramas}
\subsection{Arquitetura}

\includegraphics{arch.1}
\pagebreak

\subsection{Modelo Físico}

\smartmakeincludegraphics[width=\linewidth]{modelofisico}
\pagebreak

\subsection{Exemplo Físico}

Dois usuários ligados, um usuário sem amigos, todos possuem mensagem

\smartmakeincludegraphics[width=\linewidth]{exemplofisico}
\pagebreak

\section{Assertivas estruturais do módulo grafo}

Assertivas marcadas com $\oplus$ são executáveis.\newline
Para qualquer \texttt{struct Graph G}:
\begin{enumerate}

\item \texttt{struct Graph} (cabeça)
	\begin{enumerate}
		\item O valor de \texttt{G.nodes} é sempre o enderço de uma
			lista válida.
		\item $\oplus$ O número de \texttt{struct Node} alocados é sempre
			o número de elementos na lista \texttt{G.nodes}
		\item $\oplus$ O número de \texttt{struct Link} alocados é sempre par.
		\item $\oplus$ \texttt{G.nOfNodes != 0} $\Rightarrow$
			\texttt{ G.currentNode != NULL}
		\item \texttt{G.nOfNodes != 0} $\Rightarrow$
			\texttt{ G.currentNode } é o endereço de uma lista válida.
	\end{enumerate}

\item \texttt{struct Node} (nó)
	Para qualquer \texttt{struct Node N}:
	\begin{enumerate}
		\item O valor de \texttt{N.links} é sempre o endereço de uma lista válida.
		\item $\oplus$ Se N pertence à lista \texttt{G.nodes},
			\texttt{N.delData == G.delData}
	\end{enumerate}

\item \texttt{struct Link} (aresta)
	Para qualquer \texttt{struct Link L}:
	\begin{enumerate}
		\item $\oplus$ Existe uma \texttt{struct link B} tal que
			\begin{verbatim}
				L.brother == &B
				B.brother == &L
				&L != &B
				L.n1 == B.n2
				L.n2 == B.n1
			\end{verbatim}
		\item $\oplus$ L é um elemento da lista \texttt{L.n2->links}
		\item $\oplus$ \texttt{L.brother} é um elemento da lista
			\texttt{L.n1->links}
	\end{enumerate}
\end{enumerate}


\end{document}
